\documentclass{article}
\usepackage[left=3cm, right=3cm, top=0cm]{geometry}
\usepackage{amsmath}
\usepackage{graphicx}
\usepackage{tikz}
\usepackage{hyperref}
\usepackage[official]{eurosym}
\usepackage{fontspec}
\setlength\parindent{0pt}
\hypersetup{
    colorlinks,
    citecolor=green,
    filecolor=black,
    linkcolor=blue,
    urlcolor=blue
}

\begin{document}
\title{B166 LBA}
\author{Jacob Puthipiroj}
\date{}
\maketitle

For my LBA, I visited 9 coffee shops around Berlin. The prices collected are given in appendix. Among the cafes visited, there were two that stood out as having diametrically opposed pricing and marketing strategies: Bonanza coffee's two branches in Kreuzberg and Prenzlauer Berg, and Coffee Star in Prenzlauer Berg. I will compare the marketing efforts of both brands using the 4Ps as my analytical model.\\


Bonanza Coffee, as a brand, exudes luxury. On its website, Bonanza boasts of `Farmers, planting varietals sacrificing yield over cup quality, coffee pickers only picking red ripe cherries during harvest, importers storing coffee in climate controlled warehouses.' To enter Bonanza's Kreuzberg branch, one does not merely walk through the front, but must walk along a sideroad past two imposing doors and through a garden before finding a big green gate opening up to a giant, bright space. Because of its location in the relatively cheap neighborhood of Kreuzberg, seems an almost deliberate attempt to juxtapose an oasis of luxury amongst a sea of lackluster streets-side cafes. By the time a customer has walked across the marble floors and sat at table along the oak walls, a 3.5 euro Cappuccino seems entirely justifiable.\\

Miles away in Prenzlauer Berg, one of the richest boroughs in Berlin, is Bonanza's other branch. Here, prices are lower compared to in Kreuzberg. A flat white costs \EUR3.2 rather than \EUR3.4, and a cappuccino \EUR3.2 rather than \EUR3.6. Bonanza seems locked in fierce competition with other high-end cafes. Yet in Prenzlauer Berg, where young urban professionals frequent sophisticated restaurants and expensive coffee houses, Coffee Star, a low end cafe, rather than Bonanza, stands out.\\

Coffee Star is always packed with customers as ethnically diverse as its music. Most customers leave moments after their purchase; there is no WiFi and only two tables to keep them. The main attraction are the fair trade espressos for \EUR0.95 per cup. Coffee Star easily sells one of the cheapest coffees in Berlin, and in Prenzlauer Berg,  penetration pricing is used to full effect as it costs half or even a third of its competitors. Coffee Star's business model is designed to maximize revenue through quantity. Its website is available only in German, clearly targeted to a local audience. When I asked a manger if they would increase their prices by \EUR0.5, he responded, "we could, but when we keep the price under a euro, it provides a psychological shock to customers and they think to themselves that they will save money by coming here often". Implicitly, it seems Coffee Star is playing the long game by maximizing customer lifetime value through multiple inexpensive purchases, rather than a one-off large purchase.\footnote{\#opsPricing: Here I elaborate on two very different examples of how pricing decisions affect operations., In Bonanza, the choice of a premium pricing strategy demands that they invest in a long and expensive supply chain stretching from the planting and harvesting to warehousing and packaging. Coffee Star, on the other hand, by adopting a low pricing strategy, must maximize efficiency and minimize customer time spent in the restaurant.}\\

Bonanza also attempts to maximize customer lifetime value through brand loyalty building. According to Johann, the Kreuzberg branch manager, over 60\% of Bonanza's revenue comes from its complementary products and services, not its coffee. These include hoodies (\EUR89), scales (\EUR167-\EUR274, depending on the color), barista lessons for corporate events, and even consultations for other coffee houses in Berlin. In itself, these products include various pricing strategies such as decoy pricing. Its coffee beans are artisanal and imported from Ethiopia, Guatemala, El Salvador, Kenya, Indonesia, and Colombia. Yumi Choi, CEO of Bonanza proudly boasts of the pastries, which come "from the best French pastry chef in town ... it took us 10 years to convince him to sell his croissants to us" (Ernst 2017). In its promotional materials, Bonanza goes beyond selling coffee and attempts to sell a lifestyle.\footnote{\#nudge: Bonanza offers a white (\EUR167), black (\EUR190) and premium (\EUR274) scale. Whereas before customers may have regarded the premium scale as the better but unaffordable option, the presentation of the black scale as an irrelevant and asymmetrically dominated decoy means customers are now more inclined to at least buy the white scale, since it is cheaper than the black one.} 

\newgeometry{}

Bonanza's website, before attempting to sell their coffee, will first mention their values, including "curiosity", "mindfulness" and "harmony". The aesthetic of the packaging is light, transparent and minimalist, and reminiscent of Apple.\\



Whereas Bonanza employs product differentiation to justify its price, it would seem Coffee Star is using price differentiation to justify its product. More precisely, Bonanza attempts to reduce price elasticity by attempting to using sunk costs (selling expensive coffee scales which will incentivize customers to buy more quality coffee beans), as well as branding themselves as the default premium choice once they have bought into the lifestyle. Coffee Star assumes that coffee, as a regular consumable is basically a price inelastic commodity, and does not attempt to offer any promotions and instead tries to price its coffee nearer to the demand equilibrium. Because of its understanding of Coffee as a base commodity, Coffee Star's promotes its coffee in a down to earth and relatable way.\\


The stark contrast between two highly successful coffee houses, specifically Bonanza Coffee, as an expensive cafe in a cheap neighborhood, and Coffee Star, as a cheap cafe in an expensive neighborhood speaks to the efficacy of different viable marketing strategies when accompanied by the appropriate marketing mix.\footnote{\#modelMarketing: In this essay, I have focused on analyzing the marketing efforts of Bonanza and Coffee Star through the lens of the 4Ps as my analytical model.}



\section*{References}

Ernst, T. (2017, September 07). Bonanza Coffee Adds To Kreuzberg's Rich Coffee Scene With Second Cafe. Retrieved from https://sprudge.com/bonanza-coffee-kreuzberg-122704.html









\end{document}
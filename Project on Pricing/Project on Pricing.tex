\documentclass[11pt]{article}
\usepackage[left=2cm, right=2cm, top=1cm]{geometry}
\setlength{\parindent}{0pt}
\begin{document}

\title{Project on Pricing}
\author{Jacob Puthipiroj}
\maketitle

The promises of IoT are but the latest in a long line of rapid changes to the call center industry.  Originating from the traditional `call centers' of the 1970s, the name change to the `contact center' of today recognizes an expansion of customer interaction in both scope and purpose. Scope refers specifically to the increasingly multichannel, or `omnichannel' nature of communication, whereby voice calls are supplanted by other modes, including web chat, email, video, social media. Current trends point to a 2019 in which less than half of total inbound communication will be conducted by voice (Haas 2017). Instead, proponents of the IoT economy envision a heavily integrated future, in which companies with the capability to process the incoming deluge of data will secure a strategic advantage by way of top-tier customer service.\\

Today, the current contact center landscape is abound with innovative IoT examples. Medtronic PLC offers a `telehealth' solution where patients with complex and chronic conditions are linked to multiple diagnostic IoT devices which allow real-time collection of symptom data (Daarkins 2015). By investing aggressively in data analytics, Medtronic operates a contact center capable of monitoring over 100,000 patients simultaneously (Metronic). In the case of Tesla, the contact center is directly fed real-time data from the car's sensors, and therefore always kept up-to-date with customer behavior. In September of 2018, the Tesla contact center was able to detect that a customer had suffered a severe car crash and proceeded to call the customer in under two minutes to offer assistance (Hanley). In 2017, after receiving a recall order from the NHTSA, the Tesla contact center alerted customers to simply install an over-the-air software update, thereby instantaneously fixing 29,222 vehicles remotely (Burke). Both Medtronic - which straddle the line between medical technology and data analytics - and Tesla -between car company and technology company respectively-  indicate a future economy where companies in otherwise low-tech industries increasingly seek data driven solutions to customer experience problems.\footnote{\#dataDrivenEnterprise:  I consider the decision that companies must use IoT in their business model to be able to achieve the new golden standard of proactive customer service. For the impact, the customer is greater peace of mind for both Medtronic and Tesla, and for the company, this results in a greater bottom line. A further impact, as mentioned, is a future economy with more data driven solutions to CX problems.}\\

For many companies, IoT-integrated contact centers will be the basis for competitive advantage. According to Aberdeen Group, companies that provide consistent quality across multiple channels retain 89\% of customers, while those who do not only retain 33\% (Minkara 2013). Not only will companies with IoT-enabled customer service be able to charge more for better quality experience, they will be able to conduct sales and price products through its contact center through personalized data. For example, Progressive Snapshot offers personalized automotive insurance rates by collecting data on driver behavior including distance driven, time of day, frequency of `hard braking', with the net result being a decrease in rate for 80\% of customers (Juang 2018). With greater data comes the possibility for smarter contracts and more advanced pricing strategies. With automotive insurance being very price elastic for lower income motorists (Litman 2009), Snapshot provides a win-win scheme by which drivers willing to forfeit unnecessary travel for a reduced premium will receive insurance, and Progressive will have access to low-risk customers with lower volatility on the bottom line. Specifically, a 6.5 cents per mile average insurance fee is predicted to reduce vehicle travel by 10-12\% (Litman 2009). Higher risk motorists who would naturally pay more per mile would have greater incentive to reduce mileage and thus the policy would also provide societal safety benefits.\footnote{\#opsPricing: I explore how IoT will allow for advanced and highly personalized pricing strategies using through the example of Progressive Snapshot's autoinsurance program, and how it can both increase profits for the company while also reduce costs for customers by having a better grasp on the nature of customer risk. }\footnote{\#carrotandstick: The SnapShot device, to be installed in the car, has an almost Pavlovian element to it; producing a loud beep whenever the driver makes a `hard brake', signalling to the customer that their savings are being reduced. Over time, customers are conditioned to avoid the hard brake by driving more safely, such as being leaving lots of room for braking and by driving only during the day.}\\

Because of IoT and its implications, several concepts surrounding contact centers must be redefined. Whereas before, `customer service' may have been limited to inbound customer requests, it must now uphold proactive outreach - helping the customer even before they know they need it - as a new golden standard. At the very least, contact center agents should be able to easily access large amounts of customer data and familiarize themselves with the relevant context quickly, where before a significant part of the call would be spent on describing the problem itself. According to McKinsey research, AI will soon be able to fully resolve 30-50\% of customer issues (Berg \& Raabe). Consequently, requests will be bifurcated: into those simple enough to be resolved through convenient options like AI and self-service, and then those so difficult as to require expert human input. The `agent' itself is therefore redefined from a general query handler to a product or problem specialist (Ameyo 2018). Whereas before, agents could simply follow decision tree protocols in handling customers, agents in the age of IoT will need to be proficient with automation, work comfortably across multiple channels, and employ creativity in handling unusual contexts. Finally, performance metrics will be redefined; since queries complex enough to demand human agent resolution will require more time and resources than pre-IoT averages, average call handling time will be an insufficient KPI (Hash 2017).\\

Although much of the IoT discussion seems distant; requiring a costly overhaul of existing systems, managers can begin today by retrofitting their contact centers with a so-called `IoT layer' (De La Bastide 2018). An example is the Amazon Kindle Fire, a computer tablet launched in 2011 which became hugely popular after the 2014 introduction of Amazon Mayday, a feature whereby customers can get 24-hour tech support. The representative can not only give quicker verbal directions that supersede tedious troubleshooting manuals but even draw on the screen itself to show customers where to click (Gokey 2014). By taking a rip-and-replace approach towards legacy systems, many companies fail to capitalize on the mountains of data already available in existing infrastructure. Instead, contact centers can make the gradual shift to IoT in phases, slowly familiarizing customers to the new system while building confidence, and thus flexibly prepare the organization for a multichannel customer experience.\footnote{\#dataDrivenEnterprise: I consider the possibility for enterprises to look into a gradual migration in order to capitalize on IoT and prepare their firms for the future, rather than perceiving IoT as one expensive overhaul (which would neither make sense financially nor strategically).}\footnote{\#professionalism: Followed APA guidelines.}


\newpage
\section*{References}
Ameyo Team. (2018, September 17). Why Internet of Things (IoT) is the Future of Contact Centers. Retrieved from https://www.ameyo.com/blog/why-internet-of-things-iot-is-the-future-of-contact-centers \\

Berg, J., \& Raabe, J. (n.d.). Charting the future of customer care through a core optimization philosophy. Retrieved from https://www.mckinsey.com/business-functions/operations/our-insights/charting-the-future-of-customer-care-through-a-core-optimization-philosophy \\

Burke, K. (2017, January 20). Over-the-air updates may alter NHTSA recall policy. Retrieved from https://www.autonews.com/article/20170123/OEM11/301239815/over-the-air-updates-may-alter-nhtsa -recall-policy \\

Darkins A, Kendall S, Edmonson E, Young M, Stressel P. Reduced cost and mortality using home telehealth to promote self-management of complex chronic conditions: a retrospective matched cohort study of 4,999 veteran patients. Telemed J E Health 2015 Jan; 21(1):70-6.\\

De La Bastide, D. (2018, February 20). IoT And Call Centers: The Current State Of Customer Interaction. Retrieved from https://www.freespee.com/blog/2018/02/20/iot-and-call-centers/ \\

Gokey, M. (2014, June 13). Amazon's Mayday Button Helps with Proposals and Angry Birds. Retrieved from https://www.digitaltrends.com/mobile/amazon-mayday-button-interesting-uses/ \\

Haas, A. (2017, November 06). 2017 global contact center survey | Deloitte US. Retrieved from https://www2.deloitte.com/us/en/pages/operations/articles/global-contact-center-survey.html \\

Hanley, S. (2018, September 01). Tesla Reaches Out To Model 3 Crash Victim Within Minutes. Retrieved from https://cleantechnica.com/2018/09/01/tesla-reaches-out-to-model-3-crash-victim-within-minutes/ \\

Hash, S. (2017, August 11). Hot Trends Impacting Contact Centers: IoT-enabled Service Strategies. Retrieved from https://blog.contactcenterpipeline.com/2017/08/hot-trends-impacting-contact-centers-iot - enabled-service-strategies/ \\

Juang, M. (2018, October 06). A new kind of auto insurance can lead to lower premiums, but it tracks your every move. Retrieved from https://www.cnbc.com/2018/10/05/new-kind-of-auto-insurance-can-be-cheaper-but-tracks-your-every-move.html \\

Litman, T. (2009). Transportation elasticities: How prices and other factors affect travel behavior.\\

Medtronic. (n.d.). Data Sharing in Healthcare - Leverage Actionable Insights. Retrieved from \\https://www.medtronic.com/us-en/transforming-healthcare/aligning-value/collaboration-in-healthcare/ \\leveraging-actionable-data.html \\

Minkara, O. (2013, October). Omni-Channel Customer Care: Empowered Customers Demand a Seamless Experience. Retrieved from https://www.slideshare.net/apereyra1965/executive-customer-contact-exchange \\


\end{document}